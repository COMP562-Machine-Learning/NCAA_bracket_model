% Options for packages loaded elsewhere
\PassOptionsToPackage{unicode}{hyperref}
\PassOptionsToPackage{hyphens}{url}
%
\documentclass[
]{article}
\usepackage{lmodern}
\usepackage{amssymb,amsmath}
\usepackage{ifxetex,ifluatex}
\ifnum 0\ifxetex 1\fi\ifluatex 1\fi=0 % if pdftex
  \usepackage[T1]{fontenc}
  \usepackage[utf8]{inputenc}
  \usepackage{textcomp} % provide euro and other symbols
\else % if luatex or xetex
  \usepackage{unicode-math}
  \defaultfontfeatures{Scale=MatchLowercase}
  \defaultfontfeatures[\rmfamily]{Ligatures=TeX,Scale=1}
\fi
% Use upquote if available, for straight quotes in verbatim environments
\IfFileExists{upquote.sty}{\usepackage{upquote}}{}
\IfFileExists{microtype.sty}{% use microtype if available
  \usepackage[]{microtype}
  \UseMicrotypeSet[protrusion]{basicmath} % disable protrusion for tt fonts
}{}
\makeatletter
\@ifundefined{KOMAClassName}{% if non-KOMA class
  \IfFileExists{parskip.sty}{%
    \usepackage{parskip}
  }{% else
    \setlength{\parindent}{0pt}
    \setlength{\parskip}{6pt plus 2pt minus 1pt}}
}{% if KOMA class
  \KOMAoptions{parskip=half}}
\makeatother
\usepackage{xcolor}
\IfFileExists{xurl.sty}{\usepackage{xurl}}{} % add URL line breaks if available
\IfFileExists{bookmark.sty}{\usepackage{bookmark}}{\usepackage{hyperref}}
\hypersetup{
  hidelinks,
  pdfcreator={LaTeX via pandoc}}
\urlstyle{same} % disable monospaced font for URLs
\usepackage{graphicx}
\makeatletter
\def\maxwidth{\ifdim\Gin@nat@width>\linewidth\linewidth\else\Gin@nat@width\fi}
\def\maxheight{\ifdim\Gin@nat@height>\textheight\textheight\else\Gin@nat@height\fi}
\makeatother
% Scale images if necessary, so that they will not overflow the page
% margins by default, and it is still possible to overwrite the defaults
% using explicit options in \includegraphics[width, height, ...]{}
\setkeys{Gin}{width=\maxwidth,height=\maxheight,keepaspectratio}
% Set default figure placement to htbp
\makeatletter
\def\fps@figure{htbp}
\makeatother
\setlength{\emergencystretch}{3em} % prevent overfull lines
\providecommand{\tightlist}{%
  \setlength{\itemsep}{0pt}\setlength{\parskip}{0pt}}
\setcounter{secnumdepth}{-\maxdimen} % remove section numbering

\author{}
\date{}

\begin{document}

{Winning It All }

{Winning It All: A Model to Predict the Winner of NCAA® March Madness }

{By Marigrace Seaton, Chase Wang, Luke Wheeler, and Matthew Wheeler}

{}

\hypertarget{h.qstvyj6ok5wp}{%
\subsubsection{\texorpdfstring{{The Problem
}}{The Problem }}\label{h.qstvyj6ok5wp}}

{It is hard to not be a basketball fan at the University of North
Carolina. Basketball culture is all around us. Thus, it was an easy
decision for us to create a model to predict, given regular-season game
statistics from NCAA® men's basketball, who the overall winner of the
postseason tournament would be. }

\hypertarget{h.580466xqskox}{%
\subsubsection{\texorpdfstring{{Related
Works}}{Related Works}}\label{h.580466xqskox}}

{Works such as ``Logistic Regression on Tournament
seeds''}\textsuperscript{\protect\hyperlink{ftnt1}{{[}1{]}}}{~and ``The
Tale of
Kaglerella''}\textsuperscript{\protect\hyperlink{ftnt2}{{[}2{]}}}{~take
on a similar problem. However, we found that ``Logistic Regression on
Tournament seeds'' left too much room for error, because higher-seeded
teams often lose out in the early stages of the tournament, and upsets
are a common occurrence. Similarly, ``The Tale of Kaglerella'' failed to
take several ratios into account which we found to be important in
determining team strength. }

\hypertarget{h.tsxh7upt40eq}{%
\subsubsection[{Feature Selection }]{\texorpdfstring{{Feature Selection
}{\protect\includegraphics{images/image4.png}}}{Feature Selection }}\label{h.tsxh7upt40eq}}

{We derived our training datasets from the data provided for the 2019
Kaggle Google Cloud \& NCAA® ML Competition 2019 for men's
basketball}\textsuperscript{\protect\hyperlink{ftnt3}{{[}3{]}}}{. These
datasets include information on regular-season statistics, tournament
statistics, home or away statistics, seed information and more. Using
pandas, we merged seed information and tournament outcome into the
regular season statics to create one single data frame from which we
would select our features. The challenge with preparing the data to
become features was picking what we thought to be the most important
statistics from those provided and manipulating them so that they might
be more useful and simultaneously more compact. }

{}

{We saw through a correlation heatmap (previous page) that there was a
strong correlation between a given team's number of assists, turnovers,
field goals scored, free throws scored, and whether either of the teams
was the winner of that season's tournament (indicated by a ``1'' under
either ``WisWinner'' or ``LisWinner'' for each season). }

{}

{According to Breakthrough
Basketball}\textsuperscript{\protect\hyperlink{ftnt4}{{[}4{]}}}{, the
most important statistics that affect the strength of a college
basketball team and the outcome of their future games are the
assist-to-turnover ratio, free-throw accuracy percentage, and field goal
accuracy percentage. We chose to use all three of these as features in
our model to predict the strongest teams going into the postseason
tournament. }

{\includegraphics{images/image1.png}}

{Because these statistics were not provided to us, we manipulated our
dataset using pandas dataframe to include the following for each regular
season game: the winner's and loser's assist-to-turnover ratios, the
winner's and loser's free throw accuracy percentage, and the winner's
and loser's field goal accuracy percentage. We also reduced the dataset
(right) to include only the aforementioned statistics, along with the
following: season year (indicated by ``Season''), the ID number for the
winning and losing teams (indicated by ``WTeamID'' and ``LTeamID,''
respectively), the winner's and loser's score (indicated by ``WScore''
and ``LScore,'' respectively), the location of the game (indicated by
``WLoc,'' defined as ``N'' if the game was played on a neutral court,
``A'' if the game was played on the losing team's court, and ``H'' if
the game was played on the winning team's court), whether the winner or
loser went on to win that season's tournament (0 if the team in question
did not win the tournament and 1 if it did), and the designated seeds of
the winning and losing teams (``W,'' ``X,'' ``Y,'' and ``Z'' before the
seed number referring to the team's position in the tournament bracket:
East, Midwest, South, and West, respectively). If seeds value is 0, that
indicates the team did not make it to the March Madness playoff. }

{}

{Furthermore, in order for our model to more easily predict which team
would emerge the winner, we transformed the data such that each column
of our dataset would correspond with only one team's performance in a
single game, and changed the meaning of the ``isWinner'' label to
correspond to a ``1'' or a ``0,'' depending on if the team in question
had won the game for which their stats were being shown. In other words,
we essentially separated the losing team's game data and the winning
team's game data (right). }{\includegraphics{images/image3.png}}

\hypertarget{h.dw37pcycd1oi}{%
\subsubsection{\texorpdfstring{{Method
}}{Method }}\label{h.dw37pcycd1oi}}

{We used a logistic regression model to classify teams as either a
winner or a loser based on their stats for a game. ~After that, we
trained the model on one season's games, and then validated using the
rest of the season data (excluding the 2019-2020 season). We repeated
this for every season, and checked which training model had the least
amount of misclassified games. In addition, we merged all the data for
each team for the 2019-2020
season}\textsuperscript{\protect\hyperlink{ftnt5}{{[}5{]}}}{~and got the
average values. Once we had the training model, we used it on the
averaged data to find the team with the highest chance to win. The model
that ended up having the lowest error was the one based on the 2007-2008
season, which was around 26.07\% error. The other models had higher
error but no more than 2\% higher. While high, due to the somewhat
random and spontaneous nature of basketball, a perfect system is not
necessarily feasible.}

\hypertarget{h.8nrjcbfcj293}{%
\subsubsection{\texorpdfstring{{Results
}}{Results }}\label{h.8nrjcbfcj293}}

{With the 2019-2020 season data, it gave the following teams the most
likely probabilities of being classified as a ``winner'' of a game based
on their average stats for the season.~~~~~~~~}

{\includegraphics{images/image2.png}}

{The team that it predicted to win was Gonzaga, which makes sense as it
was one of the best teams in the regular season. Many of the teams
listed (such as Hofstra and Yale) were among the best in their
conferences, and almost all of them had winning ~records. North
Carolina, on the other hand, had a losing record, which shows some of
the flaws in the model - despite doing very well in many statistics that
appear to be associated with winning games, North Carolina did not win
many games.}

{}

\hypertarget{h.9n6ljspbkh0i}{%
\subsubsection{\texorpdfstring{{GitHub Repository
}}{GitHub Repository }}\label{h.9n6ljspbkh0i}}

{Our work and data can be found
}{\href{https://www.google.com/url?q=https://github.com/COMP562-Machine-Learning/NCAA_bracket_model\&sa=D\&ust=1588630893461000}{here}}{.
}

\hypertarget{h.jq7aj7tso0kj}{%
\subsubsection{\texorpdfstring{{Contributors
}}{Contributors }}\label{h.jq7aj7tso0kj}}

{Marigrace Seaton, Chase Wang - Feature Selection and Data
transformation}

{Luke Wheeler, Matthew Wheeler - Model Construction and Results Analysis
}

\begin{center}\rule{0.5\linewidth}{0.5pt}\end{center}

\protect\hyperlink{ftnt_ref1}{{[}1{]}}{~}{\href{https://www.google.com/url?q=https://www.kaggle.com/kplauritzen/notebookde27b18258\&sa=D\&ust=1588630893463000}{Logistic
Regression on Tournament seeds}}

\protect\hyperlink{ftnt_ref2}{{[}2{]}}{~}{\href{https://www.google.com/url?q=https://www.kaggle.com/iamleonie/the-tale-of-kagglerella\&sa=D\&ust=1588630893464000}{The
Tale of Kagglerella}}

\protect\hyperlink{ftnt_ref3}{{[}3{]}}{~}{\href{https://www.google.com/url?q=https://www.kaggle.com/c/mens-machine-learning-competition-2019/data\&sa=D\&ust=1588630893462000}{Google
Cloud \& NCAA® ML Competition 2019-Men's}}{~}

\protect\hyperlink{ftnt_ref4}{{[}4{]}}{~}{\href{https://www.google.com/url?q=https://www.breakthroughbasketball.com/stats/how-we-use-stats-Hagness.html\&sa=D\&ust=1588630893463000}{The
Most Important Stats To Track For Your Basketball Team - Marcus
Hagness}}{~}

\protect\hyperlink{ftnt_ref5}{{[}5{]}}{~}{\href{https://www.google.com/url?q=https://www.kaggle.com/c/google-cloud-ncaa-march-madness-2020-division-1-mens-tournament\&sa=D\&ust=1588630893464000}{Google
Cloud \& NCAA® ML Competition 2020-NCAAM}}{~}

\end{document}
